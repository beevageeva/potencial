\documentclass[12pt]{book}
\usepackage{graphicx}
\usepackage{subfig} % make it possible to include more than one captioned figure/table in a single float
\usepackage[utf8]{inputenc}
\usepackage{hyperref}
\usepackage[intlimits]{amsmath}
\usepackage{amssymb}
\usepackage{tkz-euclide}
\usepackage{tikz}
\setlength{\oddsidemargin}{15.5pt} 
\setlength{\evensidemargin}{15.5pt}
\pretolerance=2000
\tolerance=3000
\renewcommand{\figurename}{Figura}
\renewcommand{\chaptername}{Cap\'{i}tulo}
\renewcommand{\contentsname}{\'{I}ndice}
\renewcommand{\tablename}{Tabla}
\renewcommand{\bibname}{Bibliograf\'{i}a}
\renewcommand{\appendixname}{Ap\'endices}


\title{Sistema autogravitante con simetría esférica y distribución exponencial de masa}
\date{}
\begin{document}
\section*{Nociones teóricas}

Para una distribución de masa $\rho \colon \mathbb{R}^3 \to \mathbb{R}$

\begin{footnotesize}
Notaciones: 
$\vec{x} = (x_1, x_2, x_3)$ para mostrar que $x = (x_1, x_2, x_3)$ es un vector $\in \mathbb{R}^3$
Para una function  $f \colon \mathbb{R}^3 \to \mathbb{R}^3$ $\vec{f}(\vec{x}) = (f_1(\vec{x}), f_2(\vec{x}), f_3(\vec{x}) )$ (vector)
y una  funcion $g \colon \mathbb{R}^3 \to \mathbb{R}$ (scalar)
\begin{description}
\item
$\nabla g(\vec{x}) = (\frac{\partial g}{\partial x_1}(\vec{x}), \frac{\partial g}{\partial x_2}(\vec{x}), \frac{\partial g}{\partial x_3}(\vec{x}) )$ es un  vector $\in   \mathbb{R}^3$;
\item
$\nabla  \cdot  \vec{f}(\vec{x}) =  \frac{\partial f_1}{\partial x_1}(\vec{x})+ \frac{\partial f_2}{\partial x_2}(\vec{x})+ \frac{\partial f_3}{\partial x_3}(\vec{x})$ es un scalar $(\in \mathbb{R})$
\item
$\nabla^2 g(\vec{x}) = \nabla  \cdot \nabla g(\vec{x}) = \frac{\partial^2 g}{\partial^2 x_1}(\vec{x})+ \frac{\partial^2 g}{\partial^2 x_2}(\vec{x})+ \frac{\partial^2 g}{\partial^2 x_3}(\vec{x})$ scalar $(\in \mathbb{R})
$
\end{description}
\end{footnotesize}


\begin{description}
\item De la ley de Newton la fuerza gravitatoria ejercitada en una masa m = 1 situada en el punto x es:  $\vec{F}(\vec{x}) = G \int{\frac{\vec{x\prime} - \vec{x}}{|\vec{x\prime} - \vec{x}|^3}\rho(\vec{x\prime})d^3\vec{x\prime}} $ 
y despues de hacer cálculos llegamos a $ \nabla  \cdot  \vec{F}(\vec{x}) = -4\pi G \rho(\vec{x}) $
\item Definimos el potencial gravitatorio $\Phi(\vec{x}) = -G \int{\frac{\rho(\vec{x\prime})}{|\vec{x\prime} - \vec{x}|}d^3\vec{x}} $. 
Observamos que $\vec{F}(\vec{x}) = - \nabla \Phi(\vec{x}) $ 
y después de reemplazar en la ecuación de antes se obtiene la ecuación de Poisson: $\nabla^2 \Phi(\vec{x}) = 4\pi G \rho(\vec{x}) $
\item \textbf{En coordenadas esféricas} ($r,\theta,\varphi$) \textbf{con simetria esférica} 
(las funciones solo dependen de r ($=|\vec{r}|$) y no de la posición en la esfera de radio r: los angulos $\theta$ y $\varphi$ )

Las derivadas totales coinciden con las derivadas parciales $\frac {d\Phi(r)}{dr} = \frac{\partial \Phi(r)}{\partial r} $; 
$|\vec{F}(r)| = |\nabla \Phi(r)| = \frac{\partial \Phi(r)}{\partial r} $ y 
$\nabla^2 \Phi(r) = \frac{1}{r^2} \frac{\partial }{\partial r}(r^2 \frac{\partial \phi(r)}{\partial r})$
La ecuacion Poisson:$ \frac{1}{r^2} \frac{\partial }{\partial r}(r^2 \frac{\partial \phi(r)}{\partial r}) = 4\pi G \rho(r) \implies
r^2 \frac{\partial \phi(r)}{\partial r} = 4\pi G \int{r^2\rho(r)dr} + K_1 \implies
\frac{\partial \Phi(r)}{\partial r} = \frac{4 \pi G}{r^2}\int{r^2\rho(r)dr} + \frac{K_1}{r^2}\implies
\Phi(r) = 4\pi G \int{\frac{1}{r^2}(\int{r^2\rho(r)dr})dr } + K_1\int{\frac{1}{r^2}dr} + K_2
=4\pi G \int{\frac{1}{r^2}(\int{r^2\rho(r)dr})dr } + \frac{K_1}{r} + K_2, K_1, K_2 \in \mathbb{R} (el signo - con K_1)
 $
\item El módulo de la fuerza ejercitada sobre la particula debido al movimiento en una órbita circular es $ |\vec{F}(r)| = m \frac{v_c^2}{r} $ y tiene que ser igual al  módulo la fuerza gravitatoria $\frac{\partial \Phi(r)}{\partial r}$
donde $v_c$ es la velocidad circular y m se consideró = 1
$ \implies v_c^2(r) = r\frac{\partial \Phi(r)}{\partial r} \implies
v_c^2(r) = \frac{4\pi G}{r}\int{r^2\rho(r)dr} + \frac{K}{r}, K \in \mathbb{R} \implies 
v_c(r) = (\frac{4 \pi G}{r}\int{r^2\rho(r)dr} + \frac{K}{r})^{\frac{1}{2}}, K \in \mathbb{R}
$
\item La masa $M(r) = 4 \pi \int{r^2\rho(r)dr} + K, K \in \mathbb{R}$
\item Las constantes de integración se eligen de tal forma que verifiquen las condiciones de contorno:
 $\lim_{x \to +\infty}\Phi(x) = 0, v_c(0) = 0, M(0) = 0$

\item En un sistema con simetria esférica: la proyección de una función f(r) en el plano y,z (a lo largo de la línea de visión OX)es la funcción: 
$F(s) = \int_{-\infty}^\infty{f(r)dx}$ donde s es la distancia desde el centro del circulo en el plano proyectado ($s^2 = y^2 + z^2$)
\item $r^2 = x^2 + s^2$ y la simetría esférica $ \implies  F(s) =  2\lvert \int_0^\infty{f(\sqrt{x^2+s^2})dx} \rvert  $

\begin{tikzpicture}[scale=.8]

% definitions
\tkzDefPoint(0,0){O}
\tkzDefPoint(2,2){P}
\tkzDefPoint(0,2){M}
\tkzDefPoint(3.5,0){Q}
\tkzDefPoint(2.87,2){S}
\tkzDefPoint(-2.87,2){T}


\tkzDrawCircle(O,Q)
\draw [shorten >= -5cm, shorten <=-5cm] (S)--(T) ;
\tkzDrawSegments[thick](O,M)
\tkzDrawSegments[thick](O,P)
\tkzDrawPoints(O,P,Q,T,M,S)

\node at (6,2.2){line of sight};
\node at (-0.5,1){s};
\node at (0.7,0.5){r};
\node at (0.5,2.2){x};
% labels
\tkzLabelPoints(Q,T,O,M,S,P)
\end{tikzpicture}

\item Para calcular estas funciones de forma numérica hay que establecer los límites de integración y las constantes
\item Miramos el gráfico de la función:$ f(r) = r^2 \rho(r)$
\item si  f es  continua y f(0) = 0
\item  $\int{r^2 \rho(r) dr} = \int_0^r{x^2 \rho(x) dx}$
\item Miramos el gráfico de la función $f(r) = \frac{1}{r^2}  \int_0^r{\frac{1}{x^2}\rho_c  e^{-\frac{x}{r0}}dx }$
\item si f no está definida en 0 pero es continua en (0,$\infty$) y $  \lim_{x \underset{>}{\to} 0} f(x) = 0$
\item $\Phi(r) = 4 \pi G  \int_\varepsilon^r{ \frac{1}{x^2}(\int_0^x{a^2 \rho(a)da})dx} + K_2$
(elegimos $K_1 = 0$ y $K_2$  de tal manera que $\lim_{x \to +\infty}\Phi(x) = 0 $, en práctica $K_2 = -\Phi(R_{max})$) 
\item $v_c(r) = (\frac{4 \pi G }{r}\int_0^r{x^2 \rho(x)dx} )^{\frac{1}{2}}  $ (la constante de integración es 0 porque $v_c(0) = 0$)
\item $M(r) = 4 \pi \int_0^r{x^2 \rho(x)dx}$ (la constante de integración es 0 porque $M(0) = 0$) 
\end{description}

\section*{Problema de la práctica. Solución numérica}
\begin{description}
\item Hipótesis: $\rho(r) = \rho_c e^{-\frac{r}{r_0}} $
\item Determinar $\Phi(r)$, M(r),$M_p(r)$, $v_c(r)$ 

\end{description}


\begin{description}
\item Miramos los gráficos de las funciones(plotFunctions.py) y vemos que cumplen las condiciones para poner los límites de integración explicadas arriba
\begin{figure}[!ht]
 \centering
 \includegraphics[scale=0.33]{func12Plot.png}
 \caption{\emph{Graficos de las funciones}}
\end{figure}

\item $\Phi(r) = 4 \pi G \rho_c \int_\varepsilon^r{ \frac{1}{x^2}(\int_0^x{a^2 e^{-\frac{a}{r_0}}da})dx} -\Phi(R_{max})$
\item $v_c(r) = (\frac{4 \pi G \rho_c}{r}\int_0^r{x^2 e^{-\frac{x}{r_0}}dx} )^{\frac{1}{2}}  $ 
\item $M(r) = 4 \pi \rho_c \int_0^r{x^2 e^{-\frac{x}{r_0}}dx}$ 
\item La proyección de la distribución de densidad en el plano YOZ
$D_p(s) = 2 \rho_c \lvert \int_0^\infty{e^{-\frac{\sqrt{s^2 + x^2}}{r_0}} dx}  \rvert$



\end{description}





\section*{Solución analítica}
\begin{description}

\item  $\rho(r) =  \rho_c  e^{-\frac{r}{r_0}} $  
\item Integrando por partes 2 veces:
\item $\int_0^r{x^2 e^{-\frac{x}{r_0}}dx} = - r_0 \int_0^r{x^2 (e^{-\frac{x}{r_0}})\prime dx}
=-r_0( (x^2 e^{-\frac{x}{r_0}})\Big|_0^r  - 2\int_0^r{x e^{-\frac{x}{r_0}}dx})  = 
-2 r_0^2 \int_0^r{x (e^{-\frac{x}{r_0}})\prime dx} - r_0 r^2 e^{-\frac{r}{r_0}} = -2 r_0^2 ((x e^{-\frac{x}{r_0}})\Big|_0^r - \int_0^r{e^{-\frac{x}{r_0}}dx}) - r_0 r^2 e^{-\frac{r}{r_0}} = 
-2 r_0^3 e^{-\frac{x}{r_0}}\Big|_0^r -2 r_0^2 r e^{-\frac{r}{r_0}} - r_0 r^2 e^{-\frac{r}{r_0}} = 
2 r_0^3 -2 r_0^3 e^{-\frac{r}{r_0}} -2 r_0^2 r e^{-\frac{r}{r_0}} - r_0 r^2 e^{-\frac{r}{r_0}}  
= 2  r_0^3 -  r_0 e^{-\frac{r}{r_0}} (2 r_0^2  + 2 r_0 r + r^2)
$

\item $\implies \int_\varepsilon^r{ \frac{1}{x^2}(\int_0^x{y^2 e^{-\frac{y}{r_0}}dy})dx} = 
2 r_0^3 \int_\varepsilon^r{\frac{1}{x^2}dx} - \int_\varepsilon^r{\frac{ e^{-\frac{x}{r_0}} (2 r_0^3 + 2 r_0^2 x +x^2 r_0)}{x^2}dx }=
-2 r0^3 \frac{1}{x}\Big|_\varepsilon^r -  \int_\varepsilon^r{e^{-\frac{x}{r_0}} (- (-r_0 - \frac{2 r_0^2}{x}) + r_0 (-r_0 - \frac{2 r_0^2}{x})\prime) dx } = 
2 r_0^3 (\frac{1}{\varepsilon} - \frac{1}{r}) + r_0 ((e^{-\frac{x}{r_0}})\prime (-r_0 - \frac{2 r_0^2}{x})  + e^{-\frac{x}{r_0}} (-r_0 - \frac{2 r_0^2}{x})\prime )= $(integración por partes)$
=2 r_0^3 (\frac{1}{\varepsilon} - \frac{1}{r}) + r_0 (e^{-\frac{x}{r_0}} (-r_0 - \frac{2 r_0^2}{x})) \Big|_\varepsilon^r =
 r_0^2 (\frac{2 r_0}{\varepsilon} - \frac{e^{-\frac{\varepsilon}{r_0}}(\varepsilon + 2 r_0)  }{\varepsilon} + \frac{-2 r_0 + e^{-\frac{r}{r_0}} (r + 2 r_0) }{r} )
\implies \Phi(r) = 4 \pi G \rho_c r_0^2 (\frac{2 r_0}{\varepsilon} - \frac{e^{-\frac{\varepsilon}{r_0}}(\varepsilon + 2 r_0)  }{\varepsilon}
+ \frac{-2 r_0 + e^{-\frac{r}{r_0}} (r + 2 r_0) }{r} )$

\item $M(r) = 4 \pi \rho_c \int_0^r{x^2 e^{-\frac{x}{r_0}}dx} = 4 \pi \rho_c r_0 ( 2 r_0^2 - 2 r_0^2 e^{-\frac{r}{r_0}} - 2 r_0 r e^{-\frac{r}{r_0}} - r^2 e^{-\frac{r}{r_0}}) $

\item $v_c(r) = (\frac{4 \pi G \rho_c}{r}\int_0^r{x^2 e^{-\frac{x}{r_0}}dx} )^{\frac{1}{2}}   
= (4 \pi G \rho_c r_0 (\frac{2 r_0^2}{r} -  e^{-\frac{r}{r_0}} (2 \frac{r_0^2}{r} +  2 r_0 + r) ))^{\frac{1}{2}} $


\item Usando programas que trabajan con símbolos matematicos(Mathematica y sympy(python) -  ver sympyDens.py) producen los mismos resultados 
\item Comparación entre las soluciones del potencial, masa y velocidad obtenidas de forma numérica y analítica (exp\_plot.py):
(son iguales)

\begin{figure}[!ht]
 \centering
 \includegraphics[scale=0.7]{allNumAn.png}
 \caption{\emph{izquierda: soluciones calculadas de forma numérica, derecha: analítica}}
\end{figure}

\end{description}

\clearpage

\section*{Variación de los parámetros ($r_0$ y $\rho_c$)}

\begin{description}
\item Los gráficos se realizaron con un programa python (exp\_compare.py) 
\item Se muestran los gráficos  para $\rho_c$ en $ \{0.5 * 10^5, 10^5, 1.5 * 10^5, 2.0 * 10^5 \}  $ kg/m3 y $r_0$ en $\{0.5 , 1, 1.5  ,2 \} $ kpc. 
Todas las cantidades estan expresadas en las unidades SI: densidad kg/m3, distancia m, potencial J/kg, densidad proyectada kg/m2, velocidad m/s y se consideró la constante gravitacional $G = 6.6 * 10^{-11} m^3/(kg * s^2)$

\end{description}



\begin{figure}[!ht]
 \centering
 \includegraphics[scale=0.33]{densFinal.png}
 \caption{\emph{Densidad}}
\end{figure}


\begin{figure}[!ht]
 \centering
 \includegraphics[scale=0.33]{potFinal.png}
 \caption{\emph{Potencial}}
\end{figure}


\begin{figure}[!ht]
 \centering
 \includegraphics[scale=0.33]{vcFinal.png}
 \caption{\emph{Velocidad circular}}
\end{figure}


\begin{figure}[!ht]
 \centering
 \includegraphics[scale=0.33]{massFinal.png}
 \caption{\emph{Masa}}
\end{figure}


\begin{figure}[!ht]
 \centering
 \includegraphics[scale=0.33]{dpFinal.png}
 \caption{\emph{Densidad proyectada}}
\end{figure}

\clearpage

\section*{Comparación con el potencial isócrono}


\subsection*{Solución analítica del potencial isocrono}

\begin{description}
\item Hay 2 parámetros configurables: b(b = 0 es equivalente a una masa puntual: toda la masa en el centro )  y M (la masa total del sistema) 
 
\item $ \Phi(r) =  -\frac{G  M}  {b + \sqrt{b^2 + r^2}}  $
\item usando sympy para hacer los cálculos (sympyPot.py) (las fórmulas salen como en el libro)
\item  Se define  $a = \sqrt{b^2 + r^2}$
\item $ \rho(r) =  M  ( \frac{3(b+a)a^2 - r^2(b+3a) }{4 \pi (b+a)^3  a^3 })$		
\item $ M(r) = M\frac{r^3}{(b + \sqrt{b^2 + r^2})^2 \sqrt{b^2 + r^2}} $ 
\item $ v_c(r) =  \sqrt{(G  M  r ^ 2)/((b + a)^2 * a)} $ 
\item La densidad proyectada se calcula como antes reemplazando la funcion de densidad
\end{description}



\subsection*{Comparación}
\begin{description}
\item Elegimos $r_0$ (el parametro de escala de la distribucion exponencial)= 2kpc y $\rho_c$ = 1e+5 y ejecutamos el programa para dibujar la masa para la distribución exponencial. 
\item La masa total obtenida y misma densidad central introducidas antes se introducen como parámetros para dibujar los gráficos en el caso del potencial isócrono para poder hacer una comparación (el parámetro b se calcula : $b = (\frac{3 M}{16 \pi \rho_c})^{\frac{1}{3}}$ -  reemplazando r=0 en la fórmula de la densidad)
\end{description}

\begin{verbatim}
 python exp_plot.py   --type=m  --rmax=0.5e+21  --r0=6.16e+19 --rhoC=1e+5
 python isochrone_plot.py --rhoC=1.0e5 --type=v --mass=5.9e+65 --rmax=0.8e+22
\end{verbatim}

\begin{figure}[!ht]
 \centering
 \includegraphics[scale=0.3]{potAnComp.png}
 \caption{\emph{Potencial comparado}}
\end{figure}

\begin{figure}[!ht]
 \centering
 \includegraphics[scale=0.3]{massAnComp.png}
 \caption{\emph{Masa comparada}}
\end{figure}


\begin{figure}[!ht]
 \centering
 \includegraphics[scale=0.3]{vcAnComp.png}
 \caption{\emph{Velocidad comparada}}
\end{figure}

\begin{figure}[!ht]
 \centering
 \includegraphics[scale=0.3]{densAnComp.png}
 \caption{\emph{Densidad comparada}}
\end{figure}

\begin{figure}[!ht]
 \centering
 \includegraphics[scale=0.3]{dpAnComp.png}
 \caption{\emph{Densidad proyectada comparado}}
\end{figure}


\clearpage

\section*{Conclusiones}
\begin{description}
\item Para las 2 distribuciones cuando la densidad es casi 0 la masa encerrada en el radius correspondiente es casi toda la masa (despues de este radius la masa es casi constante), el potencial se acerca a 0 y la velocidad circular tiene el máximo
\item el parámetro de escala $r_0$ y $\rho_c$
\end{description}


\clearpage

\section*{Código}

\begin{itemize}

\item Los programas python, imagenes, pdf e incluso el tex están en el repositorio git: 
	\url{https://github.com/beevageeva/potencial} 
Allí está la descripción y ejemplos de uso.

	 
\item Por razones históricas hay otro programa python mas general  (pt.py)  
\item Se muestran los gráficos en el caso A = B = 1 ($\implies r_0= \rho_c =1$) y constantes de integración 0, el segundo gráfico está calculado con la solución analítica: ver calc\_exp.py)

\begin{verbatim}
python pt.py --type=p --test=calc_exp --k=0,-8e-10 
python pt.py --type=v --test=calc_exp 
python pt.py --type=m --test=calc_exp 

\end{verbatim}
\begin{figure}[!ht]
 \centering
 \includegraphics[scale=0.18]{ptAll.png}
 \caption{\emph{Salidas de las ejecuciones de pt.py de arriba}}
\end{figure}



\end{itemize}





\end{document}

